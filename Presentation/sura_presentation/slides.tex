\documentclass[english,handout]{beamer}
\usetheme{Copenhagen}
\usepackage{babel}
\usepackage{color}
\usepackage{tikz}
\usetikzlibrary{shapes,arrows}

%\usepackage{pgfpages}
%\setbeameroption{show notes}
%\setbeameroption{show notes on second screen=right}


\institute[IITD]{Indian Institute Of Technology, Delhi}
\author[Harsh \and Alankrit]{Harsh Gupta \and Alankrit Chona}
\title{Plagiarism Detection Tool}
\date[\today]{\today}

\begin{document}

\begin{frame}[plain]
\maketitle
%\begin{center}
\underline{Project Facilitator} \\
Huzur Saran,\\
Professor,\\
Department of Computer Science \& Engineering,\\
Indian Institute of Technology, Delhi\\
%\end{center}
\note{\textcolor{red}{Harsh}}
\end{frame}
\begin{frame}
\frametitle{Objective}
\begin{itemize}
\item{Identify similarity of code written in a variety of languages.}
\end{itemize}
\note{\textcolor{red}{Harsh\\}}
\note{Emphasise the importance of such software. In both academic and law enforcement domains}
\end{frame}

\begin{frame}
\frametitle{Activity in this Field}
\begin{block}{MOSS:Measure Of Software Similarity}
\begin{itemize}
\item{``We give an upper bound on the performance of winnowing, expressed as a \textcolor{red}{ trade-off between the number of fingerprints that must be selected and the shortest match that we are guaranteed to detect}''}
\item{``The service currently uses robust winnowing, which is more \textcolor{red}{efficient and scalable} than previous algorithms we have tried.''}
\end{itemize}
\end{block}
\note{\textcolor{red}{Harsh\\}}
\note{
    \begin{itemize}
    \item{in use for 13 years}
    \item{used by most of our professors}
    \item{uses hashing algorithm:winnowing which is more efficient and scalable}
    \item{They select some hashes to represent a document,calling it the document's fingerprint and as with any selection criterion, they lose info.'}
    \item{emphasis of this algorithm is on efficiency and scalability but we believe that our algo can take into account both the factors without losing any info}
    \end{itemize}
    }
\end{frame}

\begin{frame}
\frametitle{Activity in this Field}
\begin{block}{XPlag}
\begin{itemize}
\item{``To detect plagiarism involving \textcolor{red}{multiple languages} using intermediate program code produced by a compiler suite.''}
\item{``Relies on the components of an existing generic compiler suite.''}
\end{itemize}
\end{block}
\note{\textcolor{red}{Alankrit\\}}
\note{
    \begin{itemize}
    \item{A completely different approach to this problem}
    \item{Emphasis on the fact that the same logic can be used in different programming languages}
    \end{itemize}
    }
\end{frame}

\begin{frame}
\frametitle{Proposed Approach}
%%%%%tikx
\begin{center}
\tikzstyle{decision} = [diamond, draw, fill=blue!20, 
    text width=4.5em, text badly centered, node distance=3cm, inner sep=0pt]
\tikzstyle{block} = [rectangle, draw, fill=blue!20, 
    text width=5em, text centered, rounded corners, minimum height=4em]
\tikzstyle{line} = [draw, -latex']
\tikzstyle{cloud} = [draw, ellipse,fill=red!20, node distance=2cm,
    minimum height=2em]
\tikzstyle{incloud} = [draw, ellipse,fill=red!20, node distance=3cm,
    minimum height=2em]
\begin{tikzpicture}[node distance = 2cm, auto]
    % Place nodes
    \node [block] (fe) {Front End};
    \node [incloud, left of=fe] (input) {Input};
    \node [cloud, below of=fe] (AST) {Abstract Syntax Tree};
    \node [block, below of=AST] (identify) {Back End};
    \node [incloud, right of=identify] (output) {Output};
    % Draw edges
    \path [line] (fe) -- (AST);
    \path [line] (AST) -- (identify);
    \path [line,dashed] (input) -- (fe);
    \path [line,dashed] (identify) -- (output);
\end{tikzpicture}

\end{center}
%%%%%tikx
\note{\textcolor{red}{Alankrit\\}}
\note{
    \begin{itemize}
    \item{Input= source code for detection}
    \item{FE= scanner and CUSTOM parser for the input language}
    \item{AST= a generic AST which will be similar for all languages}
    \item{BE= the computation intensive matching algorithm}
    \item{Results: matching pairs}

    \end{itemize}
    }
\end{frame}

\begin{frame}
\frametitle{FRONT END}
\begin{itemize}
\item{Language dependent.}
\item{Takes input as the source code.}
\item{Creates an intermediate abstract syntax tree.}
\end{itemize}
\note{\textcolor{red}{Harsh\\}}
\note{
    \begin{itemize}
    \item{implementation of a lexer and a CUSTOM parser for the language}
    \item{emphasize that no significant syntactical information is lost .}
    \item{point out that this information will be used throughout the entire computation.}
    \end{itemize}
    }
\end{frame}
\begin{frame}
\frametitle{BACK END}
\begin{itemize}
\item{Independent of input language.}
\item{Uses the Longest Common Subsequence Algorithm(LCS).}
\item{Uses a similarity metric that can be configured.}
\end{itemize}
\note{\textcolor{red}{Harsh\\}}
\note{
    \begin{itemize}
    \item{while independent of the input language,syntactical information will be retained and used throughout the code}
    \item{use of LCS algo to check for similarity}
    \item{ use of LCS algo in the diff program commonly used in versioning systems like git}
    \item{the similarity metric can be configured to provide for a loose or strict metric.}
    \end{itemize}
    }
\end{frame}
\begin{frame}
\frametitle{OUR EXPERIENCE SO FAR}
\begin{itemize}
\item{Analyzed source code plagiarism dispute (Delhi High Court Case)}
\item{Problems faced:}
\begin{itemize}
\item{Unrealistic runtimes}
\item{High rate of false positives}
\end{itemize}
\end{itemize}
\note{\textcolor{red}{Alankrit\\}}
\note{
    \begin{itemize}
    \item{stress the fact that we have learnt the importance of intelligently applying LCS rather than a greedy algo}
    \item{backtracking,}
    \begin{itemize}
    \item{assuming a root and its children to be a match and then verifying correctness of assumption}
    \item{increasing tightness of similarity metric as we move down the tree}
    \end{itemize}
    \end{itemize}
    }
\end{frame}
\begin{frame}
\frametitle{A NEW APPROACH}
\begin{itemize}
\item{Use of LCS algorithm.}
\item{Retaining the syntactic structure throughout computation.}
\item{No loss of information by selection criterion (as in the case of MOSS and others).}
\item{Configurable Similarity Metric.}
\end{itemize}
\note{\textcolor{red}{Harsh\\}}
\note{
    \begin{itemize}
    \item{we like the lcs algo and wonder why people have not used it before}
    \item{striking difference from others: no loss of syntactical information to the last stage of computation}
    \item{by appling LCS intelligently we can ensure that no loss of info is incurred and runtimes are reasonable}
    \item{configurable system will allow it to be used in variety of enviroments--academic, legal etc.}
    \end{itemize}
    }
\end{frame}
\begin{frame}
\frametitle{Phases of the project}
\begin{block}{Timeline}
\begin{enumerate}
\item{\textbf{Initial Phase:}}
Development of tool for foxpro language.
\item{\textbf{Phase 1:}}
Implementation of algorithm in easily extensible manner.
\item{\textbf{Phase 2:}}
Improvement of the user-interface.
\end{enumerate}
\end{block}
\begin{block}{Budget}
No financial requirements are anticipated.
\end{block}
\end{frame}
\begin{frame}
\vfill
\begin{center}
\huge{Thank You}
\end{center}
\vfill
\end{frame}
\end{document}
